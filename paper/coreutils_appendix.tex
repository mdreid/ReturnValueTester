\section{GNU Core Utilities crashes}
\label{appendix:coreutils}
\subsection{\texttt{du}}
The coreutils program \texttt{du}, which is used to list disk usage, crashed when \texttt{malloc} returned \texttt{NULL}. Upon investigation, it was determined to have used a function \texttt{setup\_dir} (Listing \ref{lst:fts-cycle.c}) which allocates memory and returns \texttt{true} if it succeeds, and \texttt{false} otherwise.

\begin{lstlisting}[label={lst:fts-cycle.c},firstnumber=47, caption={\texttt{lib/fts-cycle.c}}, language=C]
/* Set up the cycle-detection machinery.  */

static bool
setup_dir (FTS *fts)
{
  if (fts->fts_options & (FTS_TIGHT_CYCLE_CHECK | FTS_LOGICAL))
    {
      enum { HT_INITIAL_SIZE = 31 };
      fts->fts_cycle.ht = hash_initialize (HT_INITIAL_SIZE, NULL, AD_hash,
                                           AD_compare, free);
      if (! fts->fts_cycle.ht)
        return false;
    }
  else
    {
      fts->fts_cycle.state = malloc (sizeof *fts->fts_cycle.state);
      if (! fts->fts_cycle.state)
        return false;
      cycle_check_init (fts->fts_cycle.state);
    }

  return true;
}
\end{lstlisting}

The code does not check the return value, causing a segmentation fault on \texttt{setup\_dir}'s failure.
 The fix is given in Listing \ref{lst:fts.c}.
\begin{lstlisting}[label={lst:fts.c},firstnumber=986, caption={\texttt{lib/fts.c}}, language=C]
-	setup_dir(sp);

+	if (!setup_dir(sp)) {
+		return NULL;
+	}

// one of sp's members (NULL) is later dereferenced
\end{lstlisting}

\subsection{\texttt{hostid}}
\texttt{hostid}, which prints the numeric identifier for the host, crashed when \texttt{malloc} returned \texttt{NULL}. The error was determined to be in a loop block. On each iteration, a call to \texttt{malloc} is made, but at the end of the loop, the assumption is made that each one of them succeeded. Later in the code, the \texttt{nscount} variable is out of sync with the other fields in the data structure, and a null pointer dereference can occur. The fix is given in the glibc 2.23 source code shown in listing \ref{lst:res_send.c}.

\begin{lstlisting}[firstnumber=426, label={lst:res_send.c}, caption={\texttt{resolv/res\_send.c}}, language=C]
    for (ns = 0; ns < statp->nscount; ns++) {
            EXT(statp).nssocks[ns] = -1;
            if (statp->nsaddr_list[ns].sin_family == 0)
                    continue;
            if (EXT(statp).nsaddrs[ns] == NULL)
                    EXT(statp).nsaddrs[ns] =
                        malloc(sizeof (struct sockaddr_in6));
+                   if(EXT(statp).nsaddrs[ns] != NULL) {
+                       EXT(statp).nscount++;
+                   }
            if (EXT(statp).nsaddrs[ns] != NULL)
                    memset (mempcpy(EXT(statp).nsaddrs[ns],
                                    &statp->nsaddr_list[ns],
                                    sizeof (struct sockaddr_in)),
                            '\0',
                            sizeof (struct sockaddr_in6)
                            - sizeof (struct sockaddr_in));
    }
-   EXT(statp).nscount = statp->nscount;
}
\end{lstlisting}