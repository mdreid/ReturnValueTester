A significant portion of the programs tested were considered to be ``small-scale'', which, although a somewhat arbitrary distinction, helped in reasoning about our results. Although some of the programs in this group are substantial in terms of lines of code, e.g. \texttt{grep}, they are all command line programs, and most if not all of them should be familiar to the average systems programmer.

For small-scale utilities, we wrapped the calls listed in table \ref{lst:small_scale_wrappers}. The machine used was running RedHat Enterprise Linux version 6.8 (Santiago). The program versions are given in table \ref{tab:small_scale_versions}. For each wrapped call, all of the utilities were run 1000 times in order to cover many possible call patterns.

\begin{table}[h!]
\begin{center}
\begin{tabular}{ |c|c|c| }
\hline
% http://tex.stackexchange.com/questions/254019/how-can-i-only-make-table-headers-center-and-bold
\multicolumn{1}{|c|}{\textbf{Program(s)}} & \multicolumn{1}{c|}{\textbf{Project}} & \multicolumn{1}{c|}{\textbf{Version}} \\
\hline
\em{(all)} & coreutils  & 8.25 \\ 
\hline
\em{(all)} & net-tools & 1.60 \\
\hline
ctags&GNU Emacs & 24.5.1 \\
finger&Linux NetKit  & 0.17 \\
grep & GNU grep & 2.25 \\
gunzip & gzip & 1.6 \\
gzip & gzip & 1.6 \\
ip&iproute2 & 4.3.0 \\
last & util-linux & 2.27.1 \\
make & GNU make & 4.2 \\
man & man & 2.7.5 \\
sdiff&diffutils & 3.3 \\
ssh&OpenSSH & 7.2 \\
tar&GNU tar & 1.28 \\
unzip&Info-Zip  & 6.0 \\
w&procps-ng & 3.3.10 \\
zipinfo&Info-Zip & 3.0 \\
zip&Info-Zip & 3.0 \\
\hline
\end{tabular}
\caption{Versions of the small-scale utilities tested}
\label{tab:small_scale_versions}
\end{center}
\end{table}

\PreTable
\begin{lstlisting}[captionpos=b, label={lst:small_scale_wrappers},caption={Wrapped calls for testing small-scale utilities}]
calloc  creat   execv   fork    malloc  opendir pipe    pthread_create 	realloc
close   execvp  fopen   fstat   mmap    open    poll    pthread_mutex	read    
write
\end{lstlisting}
\PostTable
