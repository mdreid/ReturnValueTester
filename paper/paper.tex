\documentclass[10pt]{article}
\usepackage[margin=1in]{geometry}
\usepackage{indentfirst}
\usepackage{adjustbox}
\usepackage{listings}
\usepackage{color}
\usepackage{graphics}
\usepackage{graphicx}
\graphicspath{ {images/} }
\usepackage[hidelinks]{hyperref}
\usepackage{rotating}
\usepackage[final]{pdfpages}
\usepackage{array}
\definecolor{dkgreen}{rgb}{0,0.6,0}
\definecolor{gray}{rgb}{0.5,0.5,0.5}
\definecolor{mauve}{rgb}{0.58,0,0.82}
\lstset{frame=tb,
	aboveskip=3mm,
	belowskip=3mm,
	showstringspaces=false,
	basicstyle={\small\ttfamily},
	numbers=none,
	numberstyle=\tiny\color{gray},
	keywordstyle=\color{blue},
	commentstyle=\color{dkgreen},
	stringstyle=\color{mauve},
	breaklines=true,
	breakatwhitespace=true,
	tabsize=3
}

% some listings should actually read "table"
% save this value for later
\newcommand{\oldlstname}{Listing}
\newcommand{\newlstname}{Table}
\newcommand{\PreTable}{\renewcommand{\lstlistingname}{\newlstname}}
\newcommand{\PostTable}{\renewcommand{\lstlistingname}{\oldlstname}}

\title{Return Value Testing of Linux Applications}
% Crash Detection and Analysis via Library Interposition in Linux

%\author{Keith Funkhouser, Malcolm Reid, Colin Samplawski}
\author{Keith Funkhouser \\ \texttt{\href{mailto:wfunkhouser@cs.wisc.edu}{wfunkhouser@cs.wisc.edu}}
\and Malcolm Reid\\ \texttt{\href{mailto:mreid3@wisc.edu}{mreid3@wisc.edu}}
\and Colin Samplawski\\ \texttt{\href{mailto:csamplawski@wisc.edu}{csamplawski@wisc.edu}}
}

\date{Fall 2016}

\begin{document}
\setlength{\baselineskip}{18pt}
\maketitle

\begin{abstract}
\setlength{\baselineskip}{18pt}
We used fuzz testing methods to investigate the robustness of various Linux applications. We developed a mechanism to intercept system and library calls and return an error return code to the calling application. We defined crashes as unintentional core dumps or hangs. Of the 88 applications tested, we encountered crashes in 31 (35.2 \%) for at least one of the 22 calls we tested. We investigated the source code for some of these crashes and provide analysis in this paper. 
\end{abstract}

\section{Introduction}
Many commonly used C library functions return some value. In most cases, these values indicate something that is relevant to the  functions operations, such as a file descriptor, number of bytes processed, or a pointer to some data. Additionally, many functions that can fail return some predefined value that indicates that an error occurred during the function's execution. Unfortunately, many programmers have the bad habit of not checking these return values for such error conditions, which can lead to unexpected behaviors. In our work, we implemented a mechanism to intercept system/library calls in order to force an error value to be returned to an application. By analyzing how common Linux applications respond (or didn't) to these errors, we were able to evaluate the robustness of these applications.

We were primarily interested in finding crashes in the applications that we tested. By a crash we mean either an unintentional core dump or a hang that could only be ended by forcing the application to terminate (usually by using by a Ctrl-C command). This is a somewhat narrower definition than what is commonly used, but it allowed us to quantify crashes in an unambiguous way. Furthermore, these two cases represent situations where an application terminated without its consent. We require the core dump to be unintentional because core dumps can occur within a programs control, such as with \texttt{abort} or \texttt{assert}. These are somewhat inelegant ways of handling errors, but it does not represent a crash as we define it. Similarly, nearly all of the applications we tested displayed aberrant output that is likely not the behavior the developers intended, but this is simply our opinion and is not something that can be meaningfully measured. Nevertheless, we do include some of the more interesting non-crash behaviors that we found. ROADMAP

It is most interesting to test for unchecked return values in both Unix utilities and large scale programs. We restricted our tests to open-source projects, so we could attempt to find the source code lines that caused the bug. We group the applications that we test into three categories: GNU core utilities, networking utilities, and large open-source projects.

\section{Related Work and Motivation}
Our work is an extension of previous work in fuzz testing. Fuzz testing is conducted by passing randomly generated input to applications in order to find bugs. It was first developed by Miller et al. [?] in 1991. They used it to test how reliable UNIX applications were by counting the number that crashed or hung when passed fuzzed input. Surprisingly, they found that 25-33\% of the applications in the UNIX versions tested crashed or hung. In the same spirit, Miller et al. \cite{bart} fuzz tested more applications in 1995. They found that, although the programs that were tested in the previous study improved in reliability, many programs still crashed or hung. This study also applied fuzzing to test for failure to properly handle error return codes. To test the \texttt{malloc} family of calls, the authors intercepted \texttt{calloc}, \texttt{malloc}, and \texttt{realloc} calls from Unix applications, and with some probability $p$, returned an error value. With probability $1-p$, their program returned the true return code. Of the 53 applications tested, 25 crashed.  We extend this work by increasing both the number of calls and the number of applications tested.  

%As described by the Free Software Foundation, “the GNU core utilities are the basic file, shell and text manipulation utilities of the GNU operating system.” [?]. We selected a representative set of utilities listed in [?]. Net-tools is a set of UNIX applications for viewing and configuring network settings [?]. They have in large part been made obsolete by the “ip” utility. We tested both the Net-tools suite and the “ip” application for unchecked return values. Finally, we decided to test large projects to investigate their reliability. 
\subsection{System Call Invocation Frequencies}
To determine what calls we should test, we used strace and ltrace to monitor library and system calls, respectively. We ran calls on small command-line utilities that we felt represented a realistic workload, given in tables \ref{lst:coreutils}, \ref{lst:net-tools}, and \ref{lst:other_utilities}. Histograms of the system call counts and library call counts are shown in figures \ref{fig:sys_counts} and \ref{fig:lib_counts}, respectively. We were unable to test some calls because they did not return any value (e.g. free). Other frequently made calls were discarded because they were not conducive to library interposition. Finally, still others were we felt were unlikely to be called by application code. \\
\begin{figure}
\centering
\includegraphics[width=\textwidth]{sys_counts}
\caption{Most frequently invoked system calls in a sample run of common utilities}
\label{fig:sys_counts}
\end{figure}

\begin{figure}
\centering
\includegraphics[width=\textwidth]{lib_counts}
\caption{Most frequently invoked library calls in a sample run of common utilities}
\label{fig:lib_counts}
\end{figure}


\section{Mechanisms}
The main mechanism needed was a way to intercept system and library calls made by an application in order to inject a bad return value. The technique also needed to be difficult to detect, lest it be circumvented, and innocuous so that it would not disrupt the normal execution of the program apart from the return value given. Furthermore, it needed to be fine-grained enough so that only calls from the  application being tested would be intercepted, lest it crash our whole system.

\subsection{Library Interposition}
Using the environment variable \texttt{LD\_PRELOAD}, a user can specify a shared library object that is loaded before any others at dynamic linking time. Importantly, functions within this shared library take precedence over any other functions of the same name, even ones found in \texttt{libc} and other standard libraries.

\subsubsection{Benefits of LD\_PRELOAD}
\texttt{LD\_PRELOAD} affords several benefits to the amateur fuzz tester. Primarily, it is extremely simple to work with, and therefore difficult to get wrong: just set an environment variable and run the (dynamically linked) binary. We took this approach in order to test a greater number of calls and utilities as quickly as possible, but given more time may have opted to use binary rewriting. Furthermore, this method enables the interception of calls many levels deep in the stack, even in other dynamically linked libraries. We perceive this to be a benefit, as programs which depend on shared libraries are taking on the risk that those libraries are written incorrectly.

\subsubsection{Limitations of LD\_PRELOAD}
However, there are several limitations to this method. Significantly, statically linked applications bypass our wrappers completely; however, we did not encounter any statically linked applications. Other approaches, such as the binary rewriting techniques used by Miller et al. \cite{bart} or a dynamic instrumentation tool such as DynInst \cite{dyninst}, circumvent this issue by modifying the binary directly, and would be necessary to expand coverage to statically linked programs.

Another limitation to our method is that it is Linux-specific. \texttt{LD\_PRELOAD} is found on many Linux distributions and some other UNIX systems, such as BSD \cite{bsd}. However, there are equivalents found on other popular systems such as the \texttt{AppInit\_DLLs} registry value on Windows \cite{dll} and the \texttt{DYLD\_INSERT\_LIBRARIES} environment variable on macOS \cite{macos}. Therefore, we feel confident that this methodology could find success when applied to applications on the most common commodity operating systems.

\subsubsection{Implementation}\label{ld_preload_implementation}
For each call of interest, we created a shared library object containing a single wrapper function with the name of the call. The wrapper simply returns an error value with probability $p$, or calls the real version of the function with probability $1-p$. In this way, we probabilistically either inject an error value, or allow the application to continue as usual. A visual representation of this mechanism is given in figure \ref{fig:ld_preload}. Each program of interest was tested with one wrapped call at a time.

This probabilistic approach is used in order to (attempt to) test all of the calls to a given function within an application. If error values were returned 100\% of the time, we would not be able to evaluate program behavior on all possible code paths. For example, if a program calls \texttt{malloc} in two locations and crashes or aborts after an error value is returned from the first, then the second call will never be reached. To account for different program structures, we tested each call for each program on a variety of error return probabilities, ranging from 1\% to 50\%. Depending on the application and the call under testing, some calls (especially ones related to resource allocation) were called many times within an application, requiring a lower probability, while others were called only a handful of times, requiring a higher probability.
\begin{figure}
  \centering
	\includegraphics[width=0.8\textwidth]{ldpreload_fig}
	\caption{Library interposition allows for system and library calls to be overriden at runtime without recompilation or modification of the binary.}
  \label{fig:ld_preload}
\end{figure}

Returning the correct value from within a wrapped function is non-trivial: infinite recursion results if one attempts to invoke the same call that is being wrapped. To avoid the recursion, the \texttt{dlsym} function can be used. \texttt{dlsym} scans through the dynamically loaded libraries for a function which has the same name as the argument passed to it. Upon finding such a function, a function handle (that is, an address) is returned. This function handle can then be used in order to call the original version of the function. An example of such a wrapper is given in listing 1.

\begin{minipage}{\linewidth} %minipage so the code is split between two pages
	\lstinputlisting[caption=\texttt{malloc} wrapper, language=C,label={lst:wrapper_example}]{sample_wrapper.c}
\end{minipage}



\subsection{Wrapper Generation}
The wrappers described in section \ref{ld_preload_implementation} (and illustrated by example in listing \ref{lst:wrapper_example}) are uniformly structured: they contain one method, the signature of which is known beforehand (as it is identical to that of the overridden call). Furthermore, nothing specific about its implementation is necessary for generating the wrapper. Thus, wrappers are good candidates for automatic code generation by inspection of the appropriate header files. For example, by examining the \texttt{stdlib.h} file and knowing a few pieces of information from the \texttt{MALLOC(3)} page, we can completely construct the aforementioned wrapper:

\begin{lstlisting}[caption=Excerpt from \texttt{/usr/include/stdlib.h}, language=C]
/* Allocate SIZE bytes of memory.  */
extern void *malloc (size_t __size) __THROW __attribute_malloc__ __wur;
\end{lstlisting}

\begin{lstlisting}[caption=Excerpt from \texttt{MALLOC(3)}]
RETURN VALUE
...On error, these functions return NULL...
\end{lstlisting}

We developed an input file which contained the call name, (error) return value, \texttt{errno} (if applicable), and the header file to be included in the wrapper (usually given at the top of the \texttt{man} page). Our script parsed this input file and generated the appropriate wrappers. The script uses the \texttt{pycparser} Python module \cite{pycparser} to traverse the AST derived from the header file.

This technique has several benefits, the primary one being that it reduces development time when wanting to make small changes to all of the wrappers. For example, for the testing described here, we want to add a ``coin flip'' to the wrapper shown above, which simply determines at random whether to return the correct value or the error value. It is trivial to do this via code generation, by simply adding a few lines to the script and re-running it to update all of the downstream wrappers.

We are unsure whether the added development cost of this approach outweighs the potential speedup in development time. It is true that the wrapper files are only infrequently modified, however when dealing with almost 20 wrappers, it becomes increasingly tedious to make changes in a manual way. At the very least, we hope that these techniques are useful for others in the testing community.

\section{Testing Small-scale Utilities}
A significant portion of the programs tested were considered to be ``small-scale'', which, although a somewhat arbitrary distinction, helped in reasoning about our results. Although some of the programs in this group are substantial in terms of lines of code, e.g. \texttt{grep}, they are all command-line programs, and most, if not all, of them should be familiar to the average systems programmer.

For small-scale utilities, we wrapped the calls listed in table \ref{lst:small_scale_wrappers}. The machine used was running RedHat Enterprise Linux version 6.8 (Santiago). The program versions are given in table \ref{tab:small_scale_versions}. For each wrapped call, all of the utilities were run 1,000 times in order to cover many possible call patterns. For each utility tested, we opted for workloads that represent common use cases (see the \texttt{full\_utils.txt} file in \cite{github} for a full list of the arguments passed to each utility).

\begin{table}[h!]
\begin{center}
\begin{tabular}{ |c|c|c| }
\hline
% http://tex.stackexchange.com/questions/254019/how-can-i-only-make-table-headers-center-and-bold
\multicolumn{1}{|c|}{\textbf{Program(s)}} & \multicolumn{1}{c|}{\textbf{Project}} & \multicolumn{1}{c|}{\textbf{Version}} \\
\hline
\em{(all)} & coreutils  & 8.25 \\ 
\em{(all)} & net-tools & 1.60 \\
ctags&GNU Emacs & 24.5.1 \\
finger&Linux NetKit  & 0.17 \\
grep & GNU grep & 2.25 \\
gunzip & gzip & 1.6 \\
gzip & gzip & 1.6 \\
ip&iproute2 & 4.3.0 \\
last & util-linux & 2.27.1 \\
make & GNU make & 4.2 \\
man & man & 2.7.5 \\
sdiff&diffutils & 3.3 \\
ssh&OpenSSH & 7.2 \\
tar&GNU tar & 1.28 \\
unzip&Info-Zip  & 6.0 \\
w&procps-ng & 3.3.10 \\
zip&Info-Zip & 3.0 \\
zipinfo&Info-Zip & 3.0 \\
\hline
\end{tabular}
\caption{Versions of the small-scale utilities tested}
\label{tab:small_scale_versions}
\end{center}
\end{table}

%calloc  creat   execv   fork    malloc  opendir pipe    pthread_create 	realloc
%close   execvp  fopen   fstat   mmap    open    poll    pthread_mutex	read    
%write

% ls ../wrappers/*.so | rev | cut -d " " -f 1 | rev | cut -d "/" -f 3 | rev | cut -d _ -f 2- | rev  | column -c 80
\begin{table}
\begin{tabular}{l}
\begin{lstlisting}
calloc                  fstat                   pthread_create
close                   malloc                  pthread_mutex_init
creat                   mmap                    read
execvp                  opendir                 realloc
execv                   open                    write
fopen                   pipe
fork                    poll
\end{lstlisting}
\end{tabular}
\caption{Wrapped calls for testing small-scale utilities}
\label{lst:small_scale_wrappers}
\end{table}

\subsection{GNU Core Utilities}
The GNU Core Utilities project (henceforth ``coreutils'') comprises over 100 programs (many of them reimplementations of the original utilities found in UNIX), which provide basic file, shell, and text manipulation. Importantly, these tools are ``expected to exist on every [GNU] operating system,'' making them some of the most commonly used applications on Linux systems \cite{coreutils}. As such, the robustness of these programs is of great importance, as they will be used by many users on a variety of systems. They also make good candidates for automated testing via shell script because of their command-line interfaces.

We hypothesized that coreutils would yield few crashes or hangs. Primarily, the project is extremely mature: in 2002, the GNU fileutils, textutils, and sh-utils projects merged into the GNU coreutils project, by which point all three projects already had a substantial development history. In the same vein, the binaries from the coreutils project are used in production environments all over the world, and therefore have undergone a significant amount of testing. Additionally, these utilities in general comprise relatively few files and lines of code; therefore, we hypothesized that they would be easier to write correctly compared to larger open-source projects.

\subsubsection{Results}
In order to assess the robustness of coreutils when given error return values from a variety of calls, we selected 51 of the coreutils programs (Table \ref{lst:coreutils}) and tested them on the aforementioned 19 calls. Of the 51 utilities that we tested, 13 (25.5\%) crashed for at least one of the wrapped calls (Table \ref{lst:coreutils}). The crashes seen were isolated to the memory allocation calls \texttt{malloc}, \texttt{calloc}, and \texttt{realloc}. 

Utilities that produced core dumps were analyzed via \texttt{gdb} in order to find the offending code. In all cases analyzed, a null pointer dereference occurred due to a failure to check the return value of either \texttt{malloc}/\texttt{calloc}/\texttt{realloc}, or a helper function that performed the memory allocation (Appendix \ref{appendix:coreutils}).

% table:
% see /test_dir/lists/gen.sh
\begin{table}[h]
\begin{tabular}{l}
\begin{lstlisting}
    base64          dirname      m  logname         sort            uniq
    basename    cm  du              ls           m  stat            unlink
    cat             echo            md5sum          sum             uptime
    chmod        m  expr            mkdir           touch           users
    cksum        mr factor          mktemp          tr           m  wc
    comm            fold        c   mv              true         m  who
c   cp              groups          nproc           truncate     m  whoami
    cut             head            printenv        tsort
    date         m  hostid          pwd             tty
  r df           m  id              seq             uname
    dircolors       link            shred           unexpand
\end{lstlisting}
\end{tabular}
\caption{GNU Core Utilities tested; those that crashed are indicated with a letter to their left (\texttt{c} = \texttt{calloc}, \texttt{m} = \texttt{malloc}, \texttt{r} = \texttt{realloc}). A total of 13/51 (25.5\%) crashed for at least one call.}
\label{lst:coreutils}
\end{table}

\subsection{\texttt{net-tools}}
The \texttt{net-tools} suite includes a number of small utilities that form the base of the networking distribution for Linux \cite{nettools}. We hypothesized that \texttt{net-tools} would also yield a significant number of crashes, as some of the utilities in it have been deprecated in favor of the \texttt{ip} utility. We tested 11 utilities and found that 4 (36.4\%) crashed (table \ref{lst:net-tools}). \texttt{malloc} was the source of all  the crashes. A more detailed investigation into the root cause of some of these crashes is given in appendix \ref{appendix:net-tools}.

\PreTable
\begin{lstlisting}[label={lst:net-tools},caption={\texttt{net-tools} utilities tested; those which crashed with \texttt{malloc} are indicated with a \texttt{m} to their left. A total of 4/11 (36.4\%) crashed.}]
m dnsdomainname m ifconfig      m netstat         route
  domainname    m ipmaddr         nisdomainname   ypdomainname
  hostname        mii-tool        rarp
\end{lstlisting}
\PostTable
\subsection{Other Utilities}
Although coreutils and \texttt{net-tools} cover a substantial portion of the Linux utilities used regularly by systems programmers, there are many more tools used frequently that do not fall under either of the two projects. We selected a sample of 16 such utilities to test (table \ref{lst:other_utilities}).

Due to the heterogeneity of the sources of these programs, we struggled to construct a hypothesis about program behavior that would be consistent across all programs. However, due to the ubiquity of many of them (e.g. \texttt{make}, \texttt{grep}), we predicted that they would be fairly hardened against bugs.

Of the 16 utilities tested, 5 (31.3\%) crashed for at least one of the wrapped calls (table \ref{lst:other_utilities}). The crashes were primarily caused by \texttt{malloc}. Details of a root cause analysis for an example crash are given in Appendix \ref{appendix:other_gnu}.

\begin{table}[h]
\begin{tabular}{l}
\begin{lstlisting}
      ctags           gzip            man             unzip
    m finger        m ip              sdiff         m w
      grep            last          m ssh             zip
      gunzip    CORc  make            tar             zipinfo
\end{lstlisting}
\end{tabular}
\caption{Other small-scale utilities tested; those that crashed are indicated with a letter to their left (\texttt{C/O/R} = \texttt{close/open/read}, \texttt{c} = \texttt{calloc}, \texttt{m} = \texttt{realloc}). A total of 5/16 (31.3\%) crashed for at least one call.}
\label{lst:other_utilities}
\end{table}

\section{Testing Large-scale Programs}
In addition to the smaller utilities discussed above, we tested a collection of large-scale open source applications. Admittedly, we do not offer a well defined distinction between small and large-scale, but we loosely define large-scale as applications that are composed of substantial code bases and represent long term projects from major vendors. We believe that these applications are an important portion of our testing suite because they demonstrate the scale that modern applications can achieve today and could not in 1995 for Miller et al. Furthermore, we expected to find more crashes in applications of these types, following the common wisdom that larger programs are generally more difficult to get correct.


One difficulty in testing large-scale applications is determining the set of behaviors to test. Applications of this size offer a large set of features and many of these features do not have analogs on other applications. For example, Firefox and Chrome offer the behavior of opening and rendering a web page, while gvim and LibreOffice offer no behaviors that are analogous. We therefore restricted our testing to behaviors that are exhibited by all of these applications. Most of our testing of large-scale applications consisted of invoking the application from the command line (with \texttt{LD\_PRELOAD} of course) and then exiting the application shortly after it started (if it did). This tests a somewhat narrow portion of the features offered by each application, but it offers behavior that is semantically equivalent across the applications, which allows for more meaningful comparability. Table 1 shows the results of our tests.

\begin{table}[h!]
\centering
\caption{Results for Testing of Large-scale Applications}
		\begin{adjustbox}{width=\textwidth}
		\begin{tabular}{|c|c|c|c|c|c|c|c|c|c|c|}
			\hline
			& Chrome & gvim & Thunderbird & Firefox & VLC & LibreOffice & VirtualBox & gcc & javac & Eclipse\\
			\hline
			\texttt{malloc} & H & & & & C &  & & C & C & C \\ \hline
			\texttt{open} & & & & & & & & & & \\ \hline
			\texttt{read} & H & & & & & & & & & \\ \hline
			\texttt{write} & H & H & H & H & H & & N/A& & & H \\ \hline
			\texttt{close} & H & & H & & & H & & & & \\ \hline
			\texttt{mmap} & & C & CH & C & H & C & & & &\\ \hline
			\texttt{pthread\_create}& H & &  & C & CH & H & N/A& N/A& & C \\ \hline
			\texttt{pthread\_mutex\_init} & H & & H & H & H & C & N/A& N/A & H &  \\ \hline
			\texttt{fork} & & & & & & & & N/A & & \\ \hline
			\texttt{pipe} & H & & & & & &  & N/A & &\\ \hline
			\texttt{fopen} & & C & C & & C & & & & & \\ \hline
		\end{tabular} 
		\end{adjustbox}
		Key: H = hang, C = core dump, CH = hang and core dump (in separate runs), N/A = wasn't called by the application, empty space = called by application but didn't result in a crash 
\end{table}

The results of our tests display a variety of behaviors across the applications that we tested. For the calls tested, every application (with the exception of VirtualBox) crashed for at least one call. Of the applications that crashed, every one (with the exception of gcc) crashed for more than one call. Of the 37 distinct crashes we encountered for these applications, 15 were caused by core dumps and 22 by hangs. This suggests that these applications are slightly more resilient to errors that would lead to core dumps. 

When we first started testing large-scale applications we encountered core dumps for nearly every application for many calls. There was no obvious signs that these core dumps were caused by aborts, but with some investigation we determined that this was the case. Many of these aborts were caused by the GLib library on the application's behalf. GLib is collection of low-level system libraries developed mainly for the GNOME project. These libraries implement a variety of commonly used data structures, string manipulation functions, and a thread package (built on \texttt{pthread}) \cite{glibman}. We found that for applications that are window-based, nearly all of them used GLib to aid in window management. An investigation of the GLib source \cite{glibsource} reveals that after being passed through a number of internal functions, an erroneous return value was eventually passed to the \texttt{g\_error} function, which prints an error message and aborts the program. This use of GLib helps explains why large-scale applications are more resilient to core dumps over hangs. It also displays some of the difficulty that can arise when determining if a application truly crashed during testing. Finally, it shows that one way to avoid crashes in applications is for one group to get the code right once and then share it with other developers.





\section{Anecdotes}
In the results presented so far we have used a rigid definition of failure (core dumps and hangs). This is necessary in order to meaningfully quantify our results, but it does leave out a section of softer errors that we encountered during our testing. Not presented in our results is the fact that a large number of applications behaved in ways that seemed very odd and outside of what the developers most likely intended. We include a few examples primarily because they show the types of errors that developers could find when debugging their code using our method. Furthermore, they offer some interesting behaviors that seem worth mentioning even though they are not part of our proper results. Figures 2-5 display some odd outputs for various calls.

\begin{figure}
	\caption{ld\_preload fig}
	\includegraphics[width=\textwidth]{weird_eclipse.jpg}
\end{figure}

\section{Conclusion}

\bibliographystyle{unsrt}
\bibliography{citations}

\newpage
\appendix
\section*{Appendix}
\lstset{numbers=left}
\section{GNU Core Utilities crashes}
\label{appendix:coreutils}
\subsection{\texttt{du}}
The coreutils program \texttt{du}, which is used to list disk usage, crashed when \texttt{malloc} returned an error. Upon investigation, it was determined to have used a function \texttt{setup\_dir} (listing \ref{lst:fts-cycle.c}) which allocates memory and returns \texttt{true} if it succeeds, and \texttt{false} otherwise.

\begin{lstlisting}[label={lst:fts-cycle.c},firstnumber=47, caption={\texttt{lib/fts-cycle.c}}]
/* Set up the cycle-detection machinery.  */

static bool
setup_dir (FTS *fts)
{
  if (fts->fts_options & (FTS_TIGHT_CYCLE_CHECK | FTS_LOGICAL))
    {
      enum { HT_INITIAL_SIZE = 31 };
      fts->fts_cycle.ht = hash_initialize (HT_INITIAL_SIZE, NULL, AD_hash,
                                           AD_compare, free);
      if (! fts->fts_cycle.ht)
        return false;
    }
  else
    {
      fts->fts_cycle.state = malloc (sizeof *fts->fts_cycle.state);
      if (! fts->fts_cycle.state)
        return false;
      cycle_check_init (fts->fts_cycle.state);
    }

  return true;
}
\end{lstlisting}

The code does not check the return value, producing a segmentation fault on \texttt{setup\_dir}'s failure.
 The fix is given in listing \ref{lst:fts.c}.
\begin{lstlisting}[label={lst:fts.c},firstnumber=986, caption={\texttt{lib/fts.c}}]
-	setup_dir(sp);

+	if (!setup_dir(sp)) {
+		return NULL;
+	}

// one of sp's members (NULL) is later dereferenced
\end{lstlisting}

\subsection{\texttt{hostid}}
\texttt{hostid}, which prints the numeric identifier for the host, crashed when \texttt{malloc} returned an error. The error was determined to be in a loop block. On each iteration, a call to \texttt{malloc} is made, but at the end of the loop, the assumption is made that each one of them succeeded. Later in the code, the \texttt{nscount} variable is out of sync with the other fields in the data structure, and a null pointer dereference can occur. The fix is given in the glibc 2.23 source code shown in listing \ref{lst:res_send.c}.

\begin{lstlisting}[firstnumber=426, label={lst:res_send.c}, caption={\texttt{resolv/res\_send.c}}]
    for (ns = 0; ns < statp->nscount; ns++) {
            EXT(statp).nssocks[ns] = -1;
            if (statp->nsaddr_list[ns].sin_family == 0)
                    continue;
            if (EXT(statp).nsaddrs[ns] == NULL)
                    EXT(statp).nsaddrs[ns] =
                        malloc(sizeof (struct sockaddr_in6));
+                   if(EXT(statp).nsaddrs[ns] != NULL) {
+                       EXT(statp).nscount++;
+                   }
            if (EXT(statp).nsaddrs[ns] != NULL)
                    memset (mempcpy(EXT(statp).nsaddrs[ns],
                                    &statp->nsaddr_list[ns],
                                    sizeof (struct sockaddr_in)),
                            '\0',
                            sizeof (struct sockaddr_in6)
                            - sizeof (struct sockaddr_in));
    }
-   EXT(statp).nscount = statp->nscount;
}
\end{lstlisting}
\section{\texttt{net-tools} crashes}

\lstset{numbers=left}
\begin{lstlisting}[label={lst:netstat/malloc},firstnumber=213, caption={\texttt{netstat} crashes when \texttt{malloc} returns an error. The offending code is in the \texttt{net-tools} 1.60 source code, in \texttt{lib/inet.c:213}.}]
    pn = (struct add	r *) malloc(sizeof(struct addr));
+   if (pn == NULL) {
+   	perror("netstat");
+		return (-1);
+   }
    pn->addr = *sin;
    pn->next = INET_nn;
    pn->host = host;
    pn->name = (char *) malloc(strlen(name) + 1);
+   if (pn->name == NULL) {
+   	perror("netstat");
+		return (-1);
+   }
    strcpy(pn->name, name);
\end{lstlisting}

\section{Other Utility Crashes}
\label{appendix:other_gnu}

\texttt{grep}, which allows for regular expression search, crashed when \texttt{calloc} returned \texttt{NULL}. The error was buried deep in the call stack, but was due to a failure to check the return value from \texttt{hash\_initialize}, which initializes a hash table and returns a pointer to it (or \texttt{NULL} in the case that it fails). The fix is given in Listing \ref{lst:exclude.c}.

\begin{lstlisting}[label={lst:exclude.c},firstnumber=265, caption={\texttt{lib/exclude.c}.}, language=C]
	case exclude_hash:
	  sp->v.table = hash_initialize (0, NULL,
	                                 (options & FNM_CASEFOLD) ?
	                                   string_hasher_ci
	                                   : string_hasher,
	                                 (options & FNM_CASEFOLD) ?
	                                   string_compare_ci
	                                   : string_compare,
	                                 string_free);
+	  if (!sp->v.table)
+	    abort();
                                          
\end{lstlisting}
\includepdf[scale=.8, angle=90, pagecommand=\section{Anecdote Examples}]{an.pdf}

\end{document}
