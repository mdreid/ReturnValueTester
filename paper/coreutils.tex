\subsection{GNU Core Utilities}
The GNU Core Utilities project (henceforth ``coreutils'') comprises over 100 programs (many of them reimplementations of the original utilities found in UNIX) which provide basic file, shell, and text manipulation. Importantly, these tools are ``expected to exist on every [GNU] operating system'', making them some of the most commonly used applications on Linux systems \cite{coreutils}. As such, the robustness of these programs is of great importance, as they will be used by many users on a variety of systems. They also make good candidates for automated testing via shell script because of their command line interfaces.

We hypothesized that coreutils would yield very few crashes. Primarily, the project is extremely mature: in 2002, the GNU fileutils, textutils, and sh-utils projects merged into the GNU coreutils project, by which point all three projects already had a substantial development history. In the same vein, the binaries from the coreutils project are used in production environments all over the world, and therefore have encountered significantly more testing than many other open source projects. Additionally, these utilities in general comprise a relatively small number of files and lines of code, therefore we hypothesized that they would be easier to get correct compared to larger open source projects.

\subsubsection{Results}
In order to assess the robustness of coreutils when given error return values from a variety of calls, we selected 51 of the coreutils programs (listing \ref{lst:coreutils}) and tested them on the aforementioned 19 calls (listing \ref{lst:small_scale_wrappers}). For each wrapped call, all of the utilities were run 1000 times. Of the 51 utilities that we tested, 12 (23.5\%) crashed for at least one of the wrapped calls (listing \ref{lst:coreutils_crashes}). The crashes seen were isolated to the memory allocation calls \texttt{malloc}, \texttt{calloc}, and \texttt{realloc}. 

Utilities which produced core dumps were analyzed via \texttt{gdb} in order to find the offending code. 

% table:
\iffalse
cat ~/ReturnValueTester/test\_dir/utils.txt.bak | egrep -v '^#|^$' | cut -d "|" -f 1 | uniq | sort > utils_used.txt
ls -f -a -1 ~/Downloads/coreutils-8.25/src/*.c | rev | cut -d "/" -f 1 | rev | cut -d "." -f 1 | sort > coreutils.txt 
comm utils_used.txt coreutils.txt -12 | column -c 80
\fi

% TODO: modify to be annotated with which calls crashed each utility
%  > uniq test_dir/core_utils.txt | column -c  80 | expand
\begin{minipage}{\linewidth}
\begin{lstlisting}[label={lst:coreutils},caption={GNU Core Utilities tested}]
base64          dircolors       id              pwd             tsort
basename        dirname         link            seq             tty
cat             du              logname         shred           uname
chmod           echo            ls              sort            unexpand
cksum           expr            md5sum          stat            uniq
comm            factor          mkdir           sum             unlink
cp              fold            mktemp          touch           users
cut             groups          mv              tr              wc
date            head            nproc           true            who
df              hostid          printenv        truncate        whoami
\end{lstlisting}
\end{minipage}

% ls -f -a -1 ~/ReturnValueTester/wrappers/*_wrapper.c | rev | cut -d "/" -f 1 | rev | cut -d "_" -f 1 | uniq | column -c 80
\begin{lstlisting}[label={lst:coreutils_crashes},caption={Crashes in GNU Core Utilities}]
cp (calloc)				factor (malloc/realloc)		mv (calloc)       whoami (malloc)
df (realloc)			hostid (malloc)		    	stat (malloc)      
du (malloc/calloc)	id (malloc)						wc (malloc)
expr (malloc)			logname (malloc)				who (malloc)			
\end{lstlisting}