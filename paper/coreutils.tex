\subsection{GNU Core Utilities}
The GNU Core Utilities project (henceforth ``coreutils'') comprises over 100 programs (many of them reimplementations of the original utilities found in UNIX), which provide basic file, shell, and text manipulation. Importantly, these tools are ``expected to exist on every [GNU] operating system,'' making them some of the most commonly used applications on Linux systems \cite{coreutils}. As such, the robustness of these programs is of great importance, as they will be used by many users on a variety of systems. They also make good candidates for automated testing via shell script because of their command-line interfaces.

We hypothesized that coreutils would yield very few crashes or hangs. Primarily, the project is extremely mature: in 2002, the GNU fileutils, textutils, and sh-utils projects merged into the GNU coreutils project, by which point all three projects already had a substantial development history. In the same vein, the binaries from the coreutils project are used in production environments all over the world, and therefore have undergone a significant amount of testing. Additionally, these utilities in general comprise a relatively small number of files and lines of code; therefore, we hypothesized that they would be easier to write correctly compared to larger open-source projects.

\subsubsection{Results}
In order to assess the robustness of coreutils when given error return values from a variety of calls, we selected 51 of the coreutils programs (table \ref{lst:coreutils}) and tested them on the aforementioned 19 calls. Of the 51 utilities that we tested, 13 (25.5\%) crashed for at least one of the wrapped calls (table \ref{lst:coreutils}). The crashes seen were isolated to the memory allocation calls \texttt{malloc}, \texttt{calloc}, and \texttt{realloc}. 

Utilities that produced core dumps were analyzed via \texttt{gdb} in order to find the offending code. In all cases analyzed, a null pointer dereference occurred due to a failure to check the return value of either \texttt{malloc}/\texttt{calloc}/\texttt{realloc}, or a helper function which performed the memory allocation (Appendix \ref{appendix:coreutils}).

% table:
% see /test_dir/lists/gen.sh
\begin{table}[h]
\begin{tabular}{l}
\begin{lstlisting}
    base64          dirname      m  logname         sort            uniq
    basename    cm  du              ls           m  stat            unlink
    cat             echo            md5sum          sum             uptime
    chmod        m  expr            mkdir           touch           users
    cksum        mr factor          mktemp          tr           m  wc
    comm            fold        c   mv              true         m  who
c   cp              groups          nproc           truncate     m  whoami
    cut             head            printenv        tsort
    date         m  hostid          pwd             tty
  r df           m  id              seq             uname
    dircolors       link            shred           unexpand
\end{lstlisting}
\end{tabular}
\caption{GNU Core Utilities tested; those which crashed are indicated with a letter to their left (\texttt{c} = \texttt{calloc}, \texttt{m} = \texttt{malloc}, \texttt{r} = \texttt{realloc}). A total of 13/51 (25.5\%) crashed for at least one call.}
\label{lst:coreutils}
\end{table}
