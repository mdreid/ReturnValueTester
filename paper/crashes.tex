\subsection{\texttt{du/malloc}}
\begin{itemize}
\item Return value from \texttt{setup\_dir} is not checked, which allocates memory and returns \texttt{false} in many scenarios, including if \texttt{malloc} fails
\item \href{https://github.com/coreutils/gnulib/blob/master/lib/fts.c\#L986}{https://github.com/coreutils/gnulib/blob/master/lib/fts.c\#L986}
\item e.g. \texttt{PROB = 0.3}, \texttt{SEED = 10178}
\item Fix:
\begin{lstlisting}
-	setup_dir(sp);

+	if (!setup_dir(sp)) {
+		return NULL;
+	}
\end{lstlisting}
\end{itemize}

\subsection{\texttt{hostid/malloc}}
\begin{itemize}
\item In our version of \texttt{libc}, the return value from \texttt{malloc} is not checked, but a counter is set assuming that the return value was valid. This later causes a null pointer dereference.
\item \href{https://github.com/lattera/glibc/blob/master/resolv/res\_send.c\#L453}{https://github.com/lattera/glibc/blob/master/resolv/res\_send.c\#453}
\item e.g. \texttt{PROB = 0.3}, \texttt{SEED = 11589}
\item Fix:
\begin{lstlisting}
	for (ns = 0; ns < statp->nscount; ns++) {
			...
			
	        if (EXT(statp).nsaddrs[ns] == NULL)
	                EXT(statp).nsaddrs[ns] =
	                    malloc(sizeof (struct sockaddr_in6));
+	                if(EXT(statp).nsaddrs[ns] != NULL) {
+	                	EXT(statp).nscount++;
+	                }
			...
	}
-	EXT(statp).nscount = statp->nscount;
\end{lstlisting}
\end{itemize}
