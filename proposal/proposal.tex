\documentclass[10pt]{article}
\usepackage[margin=1in]{geometry}
\usepackage{indentfirst}
\usepackage{adjustbox}
\usepackage{listings}
\usepackage{color}
\newenvironment{allintypewriter}{\ttfamily}{\par}
\definecolor{dkgreen}{rgb}{0,0.6,0}
\definecolor{gray}{rgb}{0.5,0.5,0.5}
\definecolor{mauve}{rgb}{0.58,0,0.82}

\lstset{frame=tb,
	language=C,
	aboveskip=3mm,
	belowskip=3mm,
	showstringspaces=false,
	%columns=flexible,
	basicstyle={\small\ttfamily},
	numbers=none,
	numberstyle=\tiny\color{gray},
	keywordstyle=\color{blue},
	commentstyle=\color{dkgreen},
	stringstyle=\color{mauve},
	breaklines=true,
	breakatwhitespace=true,
	tabsize=3
}
%\usepackage{graphicx}
%\usepackage{float}

\title{Final Paper Proposal}
\author{Colin Samplawski, Malcolm Reid, and Keith Funkhouser}
\date{CS 736 - Fall 2016}

\begin{document}
\maketitle
\setlength{\baselineskip}{18pt}
\section{Introduction}
Nearly every function from commonly used C libraries returns some value. In most cases, these values indicate something that is relevant to the  functions operations, such as a file descriptor, number of bytes processed, or a pointer to some data. Additionally, most functions can return some predefined value that indicates that an error occurred during the function's execution. Unfortunately, many programmers have the bad habit of not checking these return values for such error conditions. In our work, we plan to analysis a suite of commonly used Linux applications and see if any do not correctly handle errors returned by commonly used system/library calls.

\section{Motivation and Goals}
In their paper (cite), Miller et al. modified the \texttt{malloc} library function to return \texttt{NULL} with some probability indicting that memory allocation has failed. They then tested a variety of Linux utilities using this version of \texttt{malloc} to test if they correctly handled this error case. Surprisingly, the majority of these utilities did not correctly handle this case and crashed in a variety of ways. We plan to expand on this idea and test significantly more system/library calls in a similar way.

\section{Method}
We do not expect our process to be terribly difficult to implement. Our main challenge is designing a way to wrap the system/library calls of interest. We need to be intercept calls in a way that is invisible and non-distributive to the application being tested. To solve this problem, Miller et al. extracted the binary of the call of interest (in their case \texttt{malloc}) and used a binary rewriter to rename \texttt{malloc} to \texttt{\_malloc}. They then wrote a new function called \texttt{malloc} which was called by the applications. This new version returned an error value with some probability or just passed the call along to \texttt{\_malloc}. We have come up with a similar solution that we believe is slightly more elegant.

Linux systems have a built in environment variable named \texttt{LD\_PRELOAD} which allows users to load a shared library before starting an application. Most importantly, these preloaded libraries take precedence over any other libraries loaded by the application. Therefore if this preloaded library contains a function named, for example, \texttt{open}, any calls to \texttt{open} by the application will invoke the preloaded \texttt{open} and not the version found in the system library(?). This allows us to intercept any call and return an error message with some probability. In order to call the real version of \texttt{open}, the \texttt{dlsym} function is used. This function searches through dynamically loaded libraries and returns a function handle to a function whose name is given as a argument. This handle can then be used to call the original version of \texttt{open}. The code used to wrap \texttt{open} is given in code listing 1. This code is compiled into a position independent shared library file to be used with \texttt{LD\_PRELOAD}. This implementation is based on a tutorial found at (cite).
\begin{lstlisting}[caption=\texttt{open} wrapper]
#define _GNU_SOURCE	//needed to compile as PIC
#include <dlfcn.h>	//dlsym
#include <stdio.h>

//function pointer for real open
static ssize_t (*real_open) (const char *pathname, int flags) = NULL;

//open wrapper
ssize_t open(const char *pathname, int flags) {
	printf("wrapped read\n");					//do something before calling real open
	real_open = dlsym(RTLD_NEXT, "open"); 	//get addr of real opean
	real_open(pathnae, flags);					//call real open
}
\end{lstlisting}

%talking about LD_PRELOAD
	%advantages
	%compare against other methods
	%wont work on windows

\section{A Small Example}
%open call on some command line apps

\section{Schedule}

\begin{center}
	\begin{adjustbox}{width=1\textwidth}
	\def\arraystretch{1.5}
	\begin{tabular}{c |c |c |c | c | c | c}
		\hline
		Monday & Tuesday & Wednesday & Thursday & Friday & Saturday & Sunday \\
		\hline \hline
		Oct 24th & Oct 25th & Oct 26th & Oct 27th & Oct 28th & Oct 29th & Oct 30th \\ 
		 &  &  &  &  &  &  \\ \hline
		Oct 31st & Nov 1st & Nov 2nd & Nov 3rd & Nov 4th & Nov 5th & Nov 6th \\
		No class &  & No class &  & No class &  &  \\ \hline
		Nov 7th & Nov 8th & Nov 9th & Nov 10th & Nov 11th & Nov 12th & Nov 13th \\
		 &  &  &  &  &  &  \\ \hline
		Nov 14th & Nov 15th & Nov 16th & Nov 17th & Nov 18th & Nov 19th & Nov 20th \\ 
		No class &  & No class &  &  &  &  \\ \hline
		Nov 21st & Nov 22nd & Nov 23rd & Nov 24th & Nov 25th & Nov 26th & Nov 27th \\ 
		 &  &  & Thanksgiving recess & Thanksgiving recess & Thanksgiving recess & Thanksgiving recess \\ \hline
		Nov 28th & Nov 29th & Nov 30th & Dec 1st & Dec 2nd & Dec 3rd & Dec 4th \\ 
		 &  &  &  &  &  &  \\ \hline
		Dec 5th & Dec 6th & Dec 7th & Dec 8th & Dec 9th & Dec 10th & Dec 11th \\ 
		No class &  & No class&  & Paper draft due to referees &  &  \\ \hline
		Dec 12th & Dec 13th & Dec 14th & Dec 15th & Dec 16th & Dec 17th & Dec 18th \\ 
		 &  &  & Paper reviews back to author &  &  &  \\ \hline
		Dec 19th & Dec 20th & Dec 21st & Dec 22nd & Dec 23rd & Dec 24th & Dec 25th \\
		Final project papers due &  & Project Poster Session &  &  &  & \\ \hline
	\end{tabular}
	\end{adjustbox}
\end{center}

\begin{center}
	\begin{allintypewriter}
%	\begin{adjustbox}{width=1\textwidth}
	\begin{tabular}{|c |c | c|}
	\hline
	System Calls & \texttt{libc} calls & Pthread calls \\ \hline \hline
	close & free& pthread\_cond\_init\\ 
	creat & kmalloc& pthread\_create\\
	dup & malloc&pthread\_mutex\_init\\
	fork & memccpy&\\
	ioctl & printf&\\
	mkdir & strto*&\\
	mmap & &\\
 	open & &\\
	pipe & &\\
	read & &\\
	write & &\\ \hline
	\end{tabular}
	%\end{adjustbox}
	\end{allintypewriter}
\end{center}
\end{document}
