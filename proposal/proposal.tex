\documentclass[10pt]{article}
\usepackage[margin=1in]{geometry}
\usepackage{indentfirst}
\usepackage{adjustbox}
\usepackage{listings}
\usepackage{color}
\newenvironment{allintypewriter}{\ttfamily}{\par}
\definecolor{dkgreen}{rgb}{0,0.6,0}
\definecolor{gray}{rgb}{0.5,0.5,0.5}
\definecolor{mauve}{rgb}{0.58,0,0.82}

\lstset{frame=tb,
	%language=C,
	aboveskip=3mm,
	belowskip=3mm,
	showstringspaces=false,
	%columns=flexible,
	basicstyle={\small\ttfamily},
	numbers=none,
	numberstyle=\tiny\color{gray},
	keywordstyle=\color{blue},
	commentstyle=\color{dkgreen},
	stringstyle=\color{mauve},
	breaklines=true,
	breakatwhitespace=true,
	tabsize=3
}
%\usepackage{graphicx}
%\usepackage{float}

\title{Final Paper Proposal}
\author{Colin Samplawski, Malcolm Reid, and Keith Funkhouser}
\date{CS 736 - Fall 2016}

\begin{document}
\maketitle
\setlength{\baselineskip}{18pt}
\section{Introduction}
Many commonly used C library functions return some value. In most cases, these values indicate something that is relevant to the  functions operations, such as a file descriptor, number of bytes processed, or a pointer to some data. Additionally, many functions that can fail return some predefined value that indicates that an error occurred during the function's execution. Unfortunately, many programmers have the bad habit of not checking these return values for such error conditions, which can lead to unexpected behaviors. In our work, we plan to analyze a suite of standard Linux applications and see how they handle errors returned by commonly used system/library calls.

\section{Motivation and Goals}
In their paper \cite{bart}, Miller et al. modified the \texttt{malloc} library function to return \texttt{NULL} with some probability indicating that memory allocation has failed. They then tested a variety of Linux utilities using this version of \texttt{malloc} to test if the programs correctly handled this error case. Surprisingly, the majority of these utilities did not correctly handle this case and crashed in a variety of ways. This motivates us to expand on this idea and test significantly more system/library calls in a similar way.

In expanding on this work, we want to not only find crashes, but also investigate and analyze their sources. In doing so, we hope to determine where and why these calls are not being checked and whether there are any patterns. We would also like to develop a tool that others can use to test for return value errors. Likewise, programmers may use this tool as part of their test suite to confirm that their own programs handle error cases in the way expected. Table 1 displays a subset of calls which we plan to investigate.

\begin{table}
	\centering
	\caption{First round of calls to test}
	\begin{allintypewriter}
%	\begin{adjustbox}{width=1\textwidth}
	\begin{tabular}{|c |c | c|}
	\hline
	System Calls & \texttt{libc} calls & Pthread calls \\ \hline \hline
	close & free& pthread\_cond\_init\\ 
	creat & kmalloc& pthread\_create\\
	dup & malloc&pthread\_mutex\_init\\
	fork & memccpy&\\
	ioctl & printf&\\
	mkdir & strto*&\\
	mmap & &\\
 	open & &\\
	pipe & &\\
	read & &\\
	write & &\\ \hline
	\end{tabular}
	%\end{adjustbox}
	\end{allintypewriter}
\end{table}

\section{Method}
We do not expect our process to be terribly difficult to implement. Our main challenge is designing a way to wrap the system/library calls of interest. We need to be able to intercept calls in a way that is invisible and non-disruptive to the application being tested. To solve this problem, Miller et al. extracted the binary of the call of interest (in their case \texttt{malloc}) and used a binary rewriter to rename \texttt{malloc} to \texttt{\_malloc}. They then wrote a new function called \texttt{malloc} which was called by the applications being tested. This new version returned an error value with some probability or just passed the call along to \texttt{\_malloc}. We have come up with a similar solution that has a lower startup cost.

Linux systems have a built in environment variable named \texttt{LD\_PRELOAD} which allows users to load a shared library before starting an application. Most importantly, these preloaded libraries take precedence over any other libraries loaded by the application. Therefore if this preloaded library contains a function named, for example, \texttt{open}, any calls to \texttt{open} by the application will invoke the preloaded \texttt{open} and not the version found in the system library. This allows us to intercept any call and return an error message with some probability. In order to call the real version of \texttt{open}, the \texttt{dlsym} function is used. This function searches through dynamically loaded libraries and returns a function handle to a function whose name is given as a argument. This handle can then be used to call the original version of \texttt{open}. The code used to wrap \texttt{open} is given in code listing 1. This code is compiled into a position independent shared library file to be used with \texttt{LD\_PRELOAD}. This implementation is based on a tutorial found at \cite{preload}.

There are some uncertainties that arise when using \texttt{LD\_PRELOAD}. Firstly, if a function from another library calls \texttt{open} it may bypass our wrapped version. Similarly, if the library containing the method of interest is statically linked to a program, the original version may be called. We are currently investigating ways to handle these problems, and we feel that we can design an implementation using \texttt{LD\_PRELOAD} that is robust to these situations. If not, we will use a method similar to the one found in Miller et al. Another issue is that the \texttt{LD\_PRELOAD} method may not apply to other operating systems.

\subsection{Auto-Generating Wrappers}
Due to the large number of calls we are interested in testing, it is in our best interest to automate the process of generating ``wrapper'' files as quickly and easily as possible. One possible technique for doing this is to traverse the abstract syntax tree (AST) of the header files, and dynamically generate C code for each method signature of interest. Listing 2 shows a trivial example for the \texttt{open} call (we used the \texttt{pycparser} Python module for parsing the AST \cite{pycparser}). However, it may be the case that this approach only covers some subset of the calls to be tested, and manual work is still required. Issues such as variadic functions and custom \texttt{typedef}s in the method signature will have to be dealt with. However, this preliminary work suggests that this technique may have potential to significantly reduce the amount of manual work required.

\begin{lstlisting}[caption=\texttt{open} wrapper, language=C]
#define _GNU_SOURCE	//needed to compile as PIC
#include <dlfcn.h>	//dlsym
#include <stdio.h>

//function pointer for real open
static ssize_t (*real_open) (const char *pathname, int flags) = NULL;

//open wrapper
ssize_t open(const char *pathname, int flags) {
	printf("wrapped read\n");					//do something before calling real open
	real_open = dlsym(RTLD_NEXT, "open"); 	//get addr of real opean
	real_open(pathnae, flags);					//call real open
}
\end{lstlisting}
\begin{lstlisting}[caption=\texttt{gen\_open\_wrapper.py}  Generates code identical to Listing 1, language=Python]
from pycparser import c_parser, c_generator

# example method signature
text = "int open(const char *pathname, int flags);"

# parse the AST
parser = c_parser.CParser()
ast = parser.parse(text, filename='<none>')

decl_node = ast.ext[-1]
decl = decl_node.type
gen = c_generator.CGenerator()

with open('open_wrapper.c', 'w') as f:
    f.write('#define _GNU_SOURCE\n')
    f.write('#include <dlfcn.h>\n')
    f.write('#include <stdio.h>\n\n')
    f.write('static %s (*real_open) (%s) = NULL;\n\n' % (gen.visit(decl.type), ', '.join([gen.visit(param) for param in decl.args.params])))
    f.write('%s {\n' % gen.visit(decl))
    f.write('  printf("wrapped open\\n");\n')
    f.write('  real_open = dlsym(RTLD_NEXT, "open");\n')
    f.write('  return real_open(%s);\n' % ', '.join([param.name for param in decl.args.params]))
    f.write('}\n')
\end{lstlisting}

%talking about LD_PRELOAD
	%advantages
	%compare against other methods
	%wont work on windows

\section{A Small Example}
As a proof of concept, we used a wrapped call to \texttt{open} on a handful of common Unix utilities: \texttt{grep}, \texttt{gcc}, and \texttt{cat}. Our call to \texttt{open} set \texttt{errno} to \texttt{EACCES}, which indicates that permission to the file is denied. Interestingly, \texttt{grep} does not make any calls to \texttt{open}. A brief overview of the source code reveals that it calls \texttt{fopen} and \texttt{fts\_open} instead. Furthermore, within the \texttt{fopen} source code, calls to \texttt{open} have a different function signature from our wrapper and therefore bypassed it. We will have to be cognizant of this overloading in the future.

We found that \texttt{gcc} and \texttt{cat} were more straightforward. We found that \texttt{gcc} made 70 calls to \texttt{open} when compiling a simple one source file C program. When we returned errors probabilistically, we found robust handling of errors in a variety of locations. In some situations when a header file could not be opened, we received the error message ``(Permission denied) Bug not reproducible." This suggests that \texttt{gcc} reattempts to compile the source file at least once. This is an example of a behavior that we will need to further investigate. Finally, \texttt{cat} displayed the simplest behavior: it either executed successfully or terminated and reported ``Permission denied."

%open call on some command line apps

\section{Schedule}
\subsection{Important Dates}
\begin{center}
	\begin{adjustbox}{width=1\textwidth}
	\def\arraystretch{1.5}
	\begin{tabular}{c |c |c |c | c | c | c}
		\hline
		Monday & Tuesday & Wednesday & Thursday & Friday & Saturday & Sunday \\
		\hline \hline
		Oct 24th & Oct 25th & Oct 26th & Oct 27th & Oct 28th & Oct 29th & Oct 30th \\ 
		 &  &  &  &  &  &  \\ \hline
		Oct 31st & Nov 1st & Nov 2nd & Nov 3rd & Nov 4th & Nov 5th & Nov 6th \\
		No class &  & No class &  & No class &  &  \\ \hline
		Nov 7th & Nov 8th & Nov 9th & Nov 10th & Nov 11th & Nov 12th & Nov 13th \\
		 &  &  &  &  &  &  \\ \hline
		Nov 14th & Nov 15th & Nov 16th & Nov 17th & Nov 18th & Nov 19th & Nov 20th \\ 
		No class &  & No class &  &  &  &  \\ \hline
		Nov 21st & Nov 22nd & Nov 23rd & Nov 24th & Nov 25th & Nov 26th & Nov 27th \\ 
		 &  &  & Thanksgiving recess & Thanksgiving recess & Thanksgiving recess & Thanksgiving recess \\ \hline
		Nov 28th & Nov 29th & Nov 30th & Dec 1st & Dec 2nd & Dec 3rd & Dec 4th \\ 
		 &  &  &  &  &  &  \\ \hline
		Dec 5th & Dec 6th & Dec 7th & Dec 8th & Dec 9th & Dec 10th & Dec 11th \\ 
		No class &  & No class&  & \textbf{Paper draft due to referees} &  &  \\ \hline
		Dec 12th & Dec 13th & Dec 14th & Dec 15th & Dec 16th & Dec 17th & Dec 18th \\ 
		 &  & \textbf{Paper reviews back to author} &  &  &  &  \\ \hline
		Dec 19th & Dec 20th & Dec 21st & Dec 22nd & Dec 23rd & Dec 24th & Dec 25th \\
		\textbf{Final project papers due} &  & \textbf{Project Poster Session} &  &  &  & \\ \hline
	\end{tabular}
	\end{adjustbox}
\end{center}

\subsection{Weekly Work Goals}
\noindent Week 1 (10/17 - 10/23): Submit project proposal\\
Week 2 (10/24 - 10/31): Write wrapper code generation code, test one call\\
Week 3 (10/31 - 11/06): Analyze source code, investigate crashes\\
Weeks 4-6 (11/07 - 11/27): Expand scope - test more calls and more programs\\
Week 7 (11/28 - 12/4): Start writing paper, package project code as tool\\
Week 8 (12/5 - 12/11): Finish writing paper\\
Week 9 (12/12 - 12/18): Revise paper, work on poster\\

\begin{thebibliography}{9}
	\bibitem{bart}
	B.P. Miller, D. Koski, C.P. Lee, V. Maganty, R. Murthy, A. Natarajan, and J. Steidl, ``Fuzz Revisited: A Re-examination of the Reliability of UNIX Utilities and Services", \textit{Computer Sciences Technical Report \#1268}, University of Wisconsin-Madison, April 1995.
	
	\bibitem{preload}
	S. Barghi, ``How to wrap a system call (libc function) in Linux", September 2014.\\
	\texttt{http://samanbarghi.com/blog/2014/09/05/how-to-wrap-a-system-call-libc-function-in-linux/}
	
	\bibitem{pycparser}
	Bendersky, E. ``pycparser: C parser and AST generator written in Python'', 2011.\\
	\texttt{https://github.com/eliben/pycparser}
\end{thebibliography}
\end{document}
